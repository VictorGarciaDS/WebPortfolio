\section{Alternativas para las integrales de diferencias binomias}
Consideremos la siguiente integral que no satisface las condiciones de los casos \ref{caso1} ó \ref{caso2}, es decir supongamos que $\frac{w+1}{t}$ y $\frac{w+1}{t}+\frac{p}{q}$ no son enteros. Sea $k\in \mathbb{N}$ tal que $\frac{w+k}{t}\in\mathbb{Z}$.

\subsection{Descomposición}
Al igual que en el caso \ref{caso1}, definamos $z_1=(a+bx^t)^{\frac{1}{q}}$, de manera que
\begin{align}
	\int x^w (a+bx^t)^{\frac{p}{q}}&=\frac{q}{tb^{m}}\int z_1^{p+q-1}(z_1^q-a)^{\frac{w+1}{t}-1}dz_1\nonumber\\
	&=\frac{q}{tb^{m}}\int (z_1^q-a)^{\frac{1-k}{t}}z_1^{p+q-1}(z_1^q-a)^{\frac{w+k}{t}-1}dz_1\nonumber
	\intertext{Definamos $m=\frac{w+k}{t}$, $p_1=1-k$, $q_1=t$, con lo que la integral anterior se reescribe como}
	&=\frac{q}{tb^{m}}\int (z_1^q-a)^{\frac{p_1}{q_1}}z_1^{p+q-1}(z_1^q-a)^{m-1}dz_1\label{eqn:5.1.1}
	\end{align}
\subsubsection{Caso polinomial}
Supongamos que $m\geq 1$, lo cual puede lograrse eligiendo $k$ suficientemente grande, así que la ecuación \ref{eqn:5.1.1} puede reescribirse como
	\begin{align*}
	\int x^w (a+bx^t)^{\frac{p}{q}}&=\frac{q}{tb^{m}}\int (z_1^q-a)^{\frac{p_1}{q_1}}z_1^{p+q-1}\left[\sum_{i=0}^{m-1}\binom{m-1}{i}z_1^{q(m-1-i)}(-a)^i\right]dz_1\\
	&=\frac{q}{tb^{m}}\int (z_1^q-a)^{\frac{p_1}{q_1}}\left[\sum_{i=0}^{m-1}\binom{m-1}{i}z_1^{p+q(m-i)-1}(-a)^i\right]dz_1\\
	&=\sum_{i=0}^{m-1}\frac{q(-a)^i}{tb^{m}}\binom{m-1}{i}\int z_1^{p+q(m-i)-1}(z_1^q-a)^{\frac{p_1}{q_1}} dz_1
	\intertext{Para cada $i$ definamos $w_{1,i}=p+q(m-i)-1$ y $t_1=q$}
	&=\sum_{i=0}^{m-1}\frac{q(-a)^i}{tb^{m}}\binom{m-1}{i}\int z_1^{w_{1,i}}(z_1^{t_1}-a)^{\frac{p_1}{q_1}} dz_1
\end{align*}
La expresión anterior es una suma de $m$ integrales de diferencias binomias distintas. El hecho de tratar con coeficientes distintos, hace que se trate de problemas distintos al original, siendo así que cada una de las integrales $\int z_1^{w_{1,i}}(z_1^{t_1}-a)^{\frac{p_1}{q_1}} dz_1$, puede abordarse mediante distintos procesos, quedando descartado el uso de los casos \ref{caso1} y \ref{caso2}.

Para poder usar el caso \ref{caso1} en cada una de las integrales, es necesario que cada  $\frac{w_{1,i}+1}{t_1}$ sea entero, lo cual es equivalente a que $\frac{p}{q}$ sea entero, lo cual no es posible. Para usar el caso \ref{caso2}, es necesario que $\frac{w_{1,i}}{t_1}+\frac{p_1}{q_1}$ sea entero, es decir, que cada $\frac{p}{q}+\frac{w+1}{t}$ sea entero, que tampoco es posible por la hipótesis inicial.

%Supongamos que $m=1$, en este caso la ecuación \ref{eqn:5.1.1} queda expresada como
%	\begin{align*}
%		\int x^w (a+bx^t)^{\frac{p}{q}}&=\frac{q}{tb^m}\int z_1^{w_{1,m}}(z_1^{t_1}-a)^{\frac{p_1}{q_1}}dz_1
%	\end{align*}
%Análogamente a cuando $m\geq 1$, no es posible abordarse usando los casos \ref{caso1} o \ref{caso2}, aunque si es una integral distinta a la original.
\subsubsection{Caso de fracciones impropias}
Supongamos que $m<0$, entonces es posible escribir la ecuación \ref{eqn:5.1.1} como
\begin{align*}
	\int x^w (a+bx^t)^{\frac{p}{q}}&=\frac{q}{tb^{m}}\int \frac{(z_1^q-a)^{\frac{p_1}{q_1}}z_1^{p+q-1}dz_1}{(z_1^q-a)^{1-m}}\\
%	&=\frac{q}{tb^{m}}\int \frac{z_1^{p+q-1}dz_1}{(z_1^q-a)^{1-m-\frac{p_1}{q_1}}}
\end{align*}

\textcolor{red}{Mencionar que el caso general ¿se puede? hacer con integración por partes}
