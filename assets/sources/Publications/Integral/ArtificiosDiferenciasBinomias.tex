\subsection{Integrales de diferencias binomias}
Las diferencias binomias son expresiones de la forma $x^w(a+bx^t)^{\frac{p}{q}}$ con $t>0$ y $\frac{p}{q}$ irreducible.
En \cite{granville_calculo_1980} se explican los cambios de variable que suelen servir para resolver integrales de diferencias binomias que cumplen la condición \ref{caso1} o la condición \ref{caso2}.
\begin{enumerate}
	\item\label{caso1} Si $\frac{w+1}{t}\in\mathbb Z$, se realiza el cambio de variable $u=(a+bx^t)^{\frac{1}{q}}$
\end{enumerate}
Para verificar la utilidad de este cambio de variable, definamos $m=\frac{w+1}{t}$ para observar que
\begin{align*}
	\int x^w(a+bx^t)^{\frac{p}{q}}dx&=\frac{1}{t}\int x^{w-t+1}(a+bx^t)^{\frac{p}{q}}(tx^{t-1}dx)\\
	&=\frac{1}{t}\int x^{t\left(\frac{w+1}{t}-1\right)}(a+bx^t)^{\frac{p}{q}}(tx^{t-1}dx)
	\intertext{Al efectuar el cambio de variable indicado, notemos que $x^t=\frac{u^q-a}{b}$ y $qu^{q-1}du=btx^{t-1}dx$, así que}
	&=\frac{1}{t}\int \left(\frac{u^q-a}{b}\right)^{m-1}u^p\frac{qu^{q-1}}{b}du\\
	&=\frac{q}{tb^{m}}\int u^{p+q-1}(u^q-a)^{m-1}du
\end{align*}
Ya que $m-1\in \mathbb Z$, entonces todos los exponentes son enteros, entonces la integral obtenida es más sencilla que la integral inicial.
\begin{problema}[$\int \frac{x^2\:dx}{\sqrt{4+x^3}}$]\label{problema25}
	Notemos que para este caso, que también puede resolverse usando \cref{eqn:2.3.4}, tenemos que $w=2$, $t=3$, $p=-1$ y $q=2$, así que $\frac{w+1}{t}=\frac{2+1}{3}=1$, entonces podemos usar el caso \ref{caso1}. Sea $u^2=4+x^3$, con lo que $2u\:du=3x^2\:dx$, así que
	\begin{align*}
		\int \frac{x^2\:dx}{\sqrt{4+x^3}}&=\int \frac{2u\:du}{3u}\\
		&=\frac{2}{3}\int du\\
		&=\frac{2}{3}u+C\\
		&=\frac{2}{3}\sqrt{4+x^3}+C
	\end{align*}
\end{problema}
\begin{enumerate}
	\setcounter{enumi}{1}
	\item\label{caso2} Si $\frac{w+1}{t}+\frac{p}{q}\in \mathbb Z$, se realiza el cambio de variable $u=\left(\frac{a+bx^t}{x^t}\right)^{\frac{1}{q}}$
\end{enumerate}
Definamos $m=\frac{w+1}{t}+\frac{p}{q}$ y veamos que al escribir la integral de la forma
\begin{align*}
	\int x^w(a+bx^t)^{\frac{p}{q}}dx&=\int x^{w+\frac{pt}{q}+t+1}\left(\frac{1}{x^t}\right)^{\frac{p}{q}}(a+bx^t)^{\frac{p}{q}}x^{-t-1}dx\\
	&=-\frac{1}{at}\int x^{w+\frac{pt}{q}+t+1}\left(\frac{a+bx^t}{x^t}\right)^{\frac{p}{q}}(-atx^{-t-1}dx)\\
	&=-\frac{1}{at}\int x^{w+1+\frac{pt}{q}+t}\left(\frac{a+bx^t}{x^t}\right)^{\frac{p}{q}}(-atx^{-t-1}dx)\\
	&=-\frac{1}{at}\int x^{t\left(\frac{w+1}{t}+\frac{p}{q}+1\right)}\left(\frac{a+bx^t}{x^t}\right)^{\frac{p}{q}}(-atx^{-t-1}dx)
	\intertext{Al efectuar el cambio de variable, podemos ver que $x^t=\frac{a}{u^q-b}$ y $qu^{q-1}du=-atx^{-t-1}dx$, entonces}
	&=-\frac{1}{at}\int \left(\frac{a}{u^q-b}\right)^{m+1} u^pqu^{q-1}du\\
	&=-\frac{a^{m}q}{t}\int \frac{u^{p+q-1}du}{(u^q-b)^{m+1}}
\end{align*}
Como $m+1\in\mathbb{Z}$, entonces todos los exponentes son enteros, lo cual simplifica la integral respecto a la integral original.
\begin{problema}[$\int \frac{dx}{x^2(4-x^4)^{\frac{3}{4}}}$]\label{problema26}
	Para esta integral, tenemos que $w=-2$, $t=4$, $p=-3$ y $q=4$. Aunque $\frac{w+1}{t}=\frac{-2+1}{4}\not\in\mathbb{Z}$, lo cual descarta el uso del caso \ref{caso1}, tenemos que $\frac{w+1}{t}+\frac{p}{q}=\frac{-2+1}{4}-\frac{3}{4}\in\mathbb{Z}$.

	Como realizaremos el cambio de variable $u^4=4x^{-4}-1$, tenemos que $4u^3\:du=-16x^{-5}dx$. Completemos la integral al escribirla como
	\begin{align*}
		\int \frac{dx}{x^2(4-x^4)^{\frac{3}{4}}}&=-\frac{1}{16}\int \frac{-16x^{-5}dx}{x^{-3}(4-x^4)^{\frac{3}{4}}}\\
		&=-\frac{1}{16}\int \frac{-16x^{-5}dx}{\left(\frac{1}{x^4}\right)^{\frac{3}{4}}(4-x^4)^{\frac{3}{4}}}\\
		&=-\frac{1}{16}\int \frac{-16x^{-5}dx}{\left(\frac{4-x^4}{x^4}\right)^{\frac{3}{4}}}
		\intertext{Efectuando el cambio de variable, obtenemos}
		&=-\frac{1}{16}\int \frac{4u^3\:du}{u^3}\\
		&=-\frac{1}{16}\int 4\:du\\
		&=-\frac{1}{4} u+C\\
		&=-\frac{1}{4}\sqrt[4]{\frac{4-x^4}{x^4}}+C
	\end{align*}
\end{problema}
En \cref{problema25,problema26} pudimos notar que en ocasiones es posible resolver las integrales de diferencias binomias, sin embargo la idea de utilizar estos cambios de variable en específico, puede funcionar en integrales de expresiones que no son propiamente diferencias binomias.

\input{ArtificiosDiferenciasBinomiasEjercicios.tex}

\input{ArtificiosDiferenciasBinomiasProblemas.tex}