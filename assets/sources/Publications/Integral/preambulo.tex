\usepackage{libertine}
\usepackage{libertinust1math}
\usepackage[T1]{fontenc}

%\usepackage{geometry}
%\usepackage[left=0cm,top=0cm,right=0cm,bottom=0cm,marginparwidth=0mm,marginparsep=0mm,margin=0cm]{geometry}

% para o markdown
\usepackage[footnotes,definitionLists,hashEnumerators,smartEllipses, hybrid]{markdown}
  \ifdefined\mobile
    \markdownSetup{rendererPrototypes={
      link = {\href{#2}{#1}}
    }}
  \fi
\newcommand{\hash}{\#}

\usepackage{microtype}

\usepackage{tocloft} % para mudar o titulo do indice

% usado para a capa...
\usepackage{tikz}
\usetikzlibrary{positioning}
\usepackage{varwidth}
\pgfdeclarelayer{bg}    % declare background layer
\pgfdeclarelayer{front} 
\pgfsetlayers{bg,main,front}  % set the order of the layers (main is the standard layer)
\usepackage[hidelinks]{hyperref}
%\usepackage{calc}
\usetikzlibrary{calc}


% usado par ao índice
\addto\captionsportuguese{
  \renewcommand{\contentsname}
    {}% A frase que aparece no lugar de "Conteúdo" no índice
}

% usado para limitar o tamanho das imagens
\usepackage[export]{adjustbox}


%\def \titulo {Título do Artigo} 
%\def \autor {} 
%\def \tradutor {} 
%\def \url {} 
%\def \CriadorDestePDF {}

% usado para fonte decorativa
\usepackage{lettrine}
\usepackage{GoudyIn}
\renewcommand{\LettrineFontHook}{\GoudyInfamily{}}
\newcommand{\DECORAR}[3][]{\lettrine[lines=3,loversize=.115,#1]{#2}{#3}}